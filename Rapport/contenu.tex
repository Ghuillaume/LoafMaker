\chapter{Noyau de l'application (Guillaume)}

	\section{Analyse}

		Structures de données et fonctionnalités
	
		-> Listes, sous-listes, tâches
		
	\section{Fonctionnalités}
	
		ghsdkglhsd
		
		


\chapter{Prototyping (Jerome)}

	\section{Storyboard}
	
	
	\section{Paper-prototype}
	

	\section{Scénarios d'utilisation}

	
	
	\section{Évaluation du prototype}
		On a demandé à Péneau ce qu'il pensait de notre paper-prototype, il a dit OK et on a modifié 2-3 trucs :)
		


\chapter{IHM}
	
	\section{Présentation (Guillaume)}
		screens
		dire que y a des popup pour confirmer la suppression (pas faire de caca) ou pour signaler à l'utilisateur que son action n'est pas possible (modifier une tache alors que y en a pas)
	
	\section{Ergonomie (Jerome)}
		Toute la difficulté pour la réalisation de l'interface graphique d'une telle application est de permettre de proposer à l'utilisateur néophyte un rendu clair et intuitif tout en proposant des fonctionnalités plus avancées pour un {\oe}il plus expert. Pour réussir cela, nous avons listé les différentes fonctionnalités souhaitées par ces deux groupes d'utilisateurs et les moyens possibles pour y arriver.
		
		Pour cela, nous avons décider de proposer plusieurs moyens d'arriver aux mêmes fonctionnalités. Par exemple, pour la création de tâches on peut passer par le menu de l'application, le bouton (+) sur la partie droite de l'écran, le clic droit sur cette même partie ou bien le raccourcis clavier. De cette manière, l'utilisateur aura toujours une solution qui lui conviendra plus que les autres et qu'il considèrera comme plus intuitive.
		
		De plus, nous avons choisi de toujours afficher le même menu pour ne pas perdre l'utilisateur avec des \og modes \fg d'utilisation. TODO
		
		l'utilisation est facilité avec
			- plusieurs manières de faire les choses
			- des boutons-icones qui permettent de s'y retrouver plus facilement qu'avec du texte
			- etc
				
	\section{Structure des widgets (Guillaume)}
		
		
	
\chapter{Limites de l'application (TOUS)}
	


\chapter{Évaluations (TOUS)}
	
	Utilisateurs néophytes :
		Ma mère et ma soeur
		Ta mère
		
	Utilisateurs avancés :
		Pierre-Yves
		Leroux

