\chapter{Noyau de l'application (Guillaume)}

	\section{Analyse}

		Structures de données et fonctionnalités
	
		-> Listes, sous-listes, tâches
		
	\section{Fonctionnalités}
	
		ghsdkglhsd
		
		


\chapter{Prototyping (Jerome)}

	\section{Storyboard}
	
	
	\section{Paper-prototype}
	

	\section{Scénarios d'utilisation}

	
	
	\section{Évaluation du prototype}
		On a demandé à Péneau ce qu'il pensait de notre paper-prototype, il a dit OK et on a modifié 2-3 trucs :)
		


\chapter{IHM}
	
	\section{Présentation (Guillaume)}
		screens
		dire que y a des popup pour confirmer la suppression (pas faire de caca) ou pour signaler à l'utilisateur que son action n'est pas possible (modifier une tache alors que y en a pas)
	
	\section{Ergonomie (Jerome)}
		l'utilisation est facilité avec
			- plusieurs manières de faire les choses
			- des boutons-icones qui permettent de s'y retrouver plus facilement qu'avec du texte
			- etc
				
	\section{Structure des widgets (Guillaume)}
		
		
	
\chapter{Limites de l'application (TOUS)}
	


\chapter{Évaluations (TOUS)}
	
	Utilisateurs néophytes :
		Ma mère et ma soeur
		Ta mère
		
	Utilisateurs avancés :
		Pierre-Yves
		Leroux

